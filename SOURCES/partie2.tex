\chapter{Différents scénarios}

En France, la position très répressive du gouvernement sur le cannabis coûte chaque année 568 millions d’euros au contribuable (source Terra Nova). Si la classe politique est peu encline à changer les choses, des pays comme les Pays-Bas, l’Uruguay ou le Colorado, pionniers dans cette législation, montrent  que ce n’est pas la seule solution.

Comment adapter la législation française pour en finir avec la répression ? Le groupe de réflexion Terra Nova, dans un rapport publié fin 2014, présente et chiffre trois régimes possibles, décrits dans les sections suivantes.

\section{Scénario 1 : Dépénalisation de l’usage}

    Le premier scénario concerne la dépénalisation de l’usage du cannabis. Par cela, on entend l’arrêt de la poursuite et de la punition des consommateurs ; cependant, la production et la vente de cannabis resterait illégale. Plusieurs pays ont déjà adopté cette politique, parmi lesquels nos voisins portugais et espagnols.

    \paragraph{Impact financier} Changer la politique de répression pour ne cibler que les producteurs et vendeurs réduirait significativement le coût de l’action policière : en France, les seules gardes à vue sur usage de cannabis constituaient 10\% du total des gardes à vue en 2013 (source OFDT). Le groupe de réflexion estime que les dépenses publiques de répression dans ce scénario se verraient amputées de 311 millions d’euros par an, ce qui représente une baisse de 55\%. En effets, les coûts d’opération de la police se verraient presque divisés par 7, tandis que les dépenses de justice, santé et prévention devraient rester à leur niveau actuel.
    
     \paragraph{Régulation du marché} Ce régime particulier ne permettrait pas de déraciner le marché noir, puisqu’il ne serait pas concurrencé par un marché légal. L’État n’aurait donc toujours aucun moyen de contrôler le prix du chanvre, qui agit directement sur la prévalence de l’usage. De la même façon, le Trésor public ne pourrait percevoir aucun revenu sur ce marché qu’il ne reconnaîtrait pas. 

    \paragraph{Coût psychologique en baisse} D’un point de vue sanitaire, supprimer les risques pénaux encourus par un consommateur en recherche du produit pourrait effectivement réduire le coût psychologique d’acquisition de l’herbe. Une telle baisse du coût tel qu’il est perçu par le consommateur ne pourrait qu’augmenter le nombre de consommateurs et la quantité de cannabis consommé. Cependant, il est à noter que le prix du marché (monétaire) resterait probablement stable, étant donné que ce scénario ne change rien à la situation des producteurs et distributeurs.

    \paragraph{Synthèse} Dans ce scénario, Terra Nova estime à 12\% l’augmentation du nombre d’usagers quotidien, et à 16\% l’augmentation du trafic en masse de produit vendu. Ces augmentations ne sont pas négligeables, et couplées au manque de contrôle de l'État sur le marché, donnent tort à cette solution sur le long terme.
    
    Les autres scénarios traitent le problème d'une autre manière, par la \textbf{légalisation}. La section suivante argumente en faveur de cette politique.





%Répression ou légalisation : la controverse sur le cannabis se résume à deux bataillons bien tranchés, où chacun campe sur ses positions. Cette polarisation n’est pas inintéressante ; elle est infiniment plus enrichissante qu’un quelconque unanimisme. Mais elle permet, à chaque parti du conflit d’idées, d’éviter soigneusement de procéder à l’analyse structurelle d’un phénomène dynamique. 
 
\section{Pourquoi légaliser le cannabis ?}
 
La légalisation du cannabis est le principe de lever tous les interdits pénaux sur la production, le commerce et l’usage du cannabis afin d’imposer un marché légal là où un marché noir est présent. L’usager du cannabis n’est plus dans ce système un délinquant ou un malade. Il est considéré comme un individu normal, sans doute affecté d’un vice, comme le jeu ou la boisson, mais les vices ne sont pas des crimes. La société les tolère tant qu’ils ne nuisent pas à l’ordre et à la moralité publique.

C’est pourquoi, dans un système de légalisation contrôlé, l’usage est interdit dans les lieux publiques, au volant, et en tout état de cause, réservé aux majeurs. Le cannabis n’étant pas une marchandise comme les autres, il est nécessaire de supprimer les pratiques qui constituent un encouragement à le produire, le vendre, ou à le consommer. Et toutes formes de promotion des ventes sont donc strictement interdites.

Ce sont ces principes qui ont été mis en œuvres à ce jour dans les quelques pays et Etats américains ayant légalisés le cannabis : En 2015, ils se comptent sur les doigts de la main : Uruguay, Colorado, Washington, Alaska, Californie , Oregon.

Ces initiatives sont encore trop fraîches pour en tirer des conclusions définitives, mais il ne serait pas étonnant que, par un effet domino, le succès de ces expérimentations ne serve d’exemple à notre pays. Certes, le principe en droit international demeure à ce jour celui d’une interdiction générale et absolue de ce produit, à l’exception de ces usages médicaux et scientifiques. Mais à la lumière de l’échec de la prohibition, les bénéfices attendus par la légalisation pourraient aider des pays à franchir le pas. Les gains indirects en termes de santé publique et de sécurité ne doivent pas être sous-estimés.

Par ailleurs, si l’on s’en tient à une lecture purement économique, il est probable que la légalisation soit une bonne affaire. La fiscalisation de ce marché et le développement d’une industrie du cannabis récréative constituerais un important gisement d’emplois, comme l’illustre l’exemple du Colorado.

La légalisation s’impose pour des raisons sanitaires, sécuritaires et économiques. Vouloir maintenir un interdit symbolique en se prévalant de la conviction qu’il est préférable de vivre sans drogue est tout à fait respectable. Il est en revanche irresponsable d’y voir une réponse juridique opératoire au défi que portent de manière aiguë la consommation et le trafic de cannabis.
 
\section{Scénario 2 : Légalisation de l’usage et la vente du cannabis dans le cadre d’un monopole public}
 
Ce scénario se penche sur la légalisation très fortement régulé par l’Etat, afin de placer le cannabis comme un bien marchand, tel que le tabac ou l’alcool, sous un monopole public.

Les principaux principes de la légalisation contrôlée par l’Etat visent à concilier le respect des libertés individuelles du consommateur de cannabis et les intérêts légitimes de la société. C’est la voie choisie par l’Uruguay : cette option permet à l’Etat de réguler le prix pour garantir une relative stabilité de la consommation. Une question se pose, à quel prix doit-on vendre le cannabis ?

\paragraph{Deux possibilités} On imagine deux scénarios possible dans ce cadre : garder le prix inchangé ou majorer la revente ---\,la possibilité de réduire le prix par rapport à celui du marché noir est directement exclue, pouvant être prise comme une incitation à la consommation.

\paragraph{À l'Uruguayenne} L’intérêt de garder le prix inchangé est d’assécher instantanément le marché noir, puisque le consommateur achètera au même prix, sans encourir de risque judiciaire, policier ou criminel. C’est (entre autres) pour cette raison que l’Uruguay est devenu, en 2013, le premier pays à contrôler intégralement la production, la vente et la consommation de cannabis. Mais cette solution engendrerait une hausse de la consommation totale de l’ordre de 185 tonnes (65\%) (53 tonnes de plus pour les usagers quotidiens et 132 tonnes de cannabis intégralement consommé par de nouveaux utilisateurs)  et une augmentation du nombre d’usagers quotidiens de l’ordre de 262 000 personnes (45\%).

\paragraph{À la Hollandaise} Le but de la légalisation, est de contrôler le nombre de consommateurs et non de le voir augmenter. C’est pourquoi une deuxième solution a été imaginée : majorer le prix de vente, « à la hollandaise ». Cette majoration a pour but de trouver une équivalence entre les prix pratiqués par le marché noir et le marché légal. En effet, la majoration correspondra à un coût que l’usager est prêt à supporter pour ne plus prendre de risque.

L’estimation (estimation forte) effectuée par Terra Nova avance une majoration de 40\% : 20\% pour le coût d’interpellation actuel (5\% pour les usagers et 15\% pour les revendeurs) et 20\% pour le risque lié au fait de côtoyer le marché illicite. Partant d’un prix moyen de 6\euro le gramme de cannabis sur le marché noir, le prix majoré serait alors de 8\euro 40 le gramme. Cette majoration permettrait donc d’avoir une augmentation du nombre d’usagers quotidien nulle, ainsi que du tonnage consommé. Cependant, le démantèlement des réseaux clandestins se fera alors plus progressivement que dans le cadre d’un prix inchangé puisque certains consommateurs seront encore attirés par des prix plus faibles, en dépit du risque encourus.

\paragraph{Impact sur les finances publiques} En se fondant sur les mêmes données et la même méthode que dans la partie précédente (Scénario 1 : Dépénaliser le cannabis), les frais de justices et de polices dûs à la légalisation seraient nuls. Seuls les frais de santé et de prévention seraient alors pris en compte, ce qui reviendrait à une dépense publique annuelle de l’ordre de 505 millions dans le cas d’un prix inchangé et de 523 millions dans le cadre d’une majoration des prix de 40\% (nombre de consommateurs moins importants que dans la proposition de prix inchangé, donc moins de frais sociaux et médicaux).
En terme de revenus publics sur les ventes, on doit prendre en compte seulement les ventes et non les parts de cannabis consommé provenant de l’auto-culture et de dons.\cite{durand_cannabis_2016}

\renewcommand{\arraystretch}{1.2}
\begin{table}\centering
\begin{tabular}{c|cccc}
&Total&Achat&Don&Auto-culture \\ \hline
En tonnes&277&208&37&32\\
En pourcents&100 \%&75.1 \% & 13.6 \% & 11.3 \% \\
\end{tabular}
\caption{Origine du cannabis (source : Ben Lakhdar 2009)}
\end{table}

 
Dans l’hypothèse 1 (prix non taxé, 6\euro) on obtient donc un chiffre d’affaires de 2.1 milliards et dans l’hypothèse que l'État taxe la vente à 80\% (comme celle du tabac), un revenu fiscal de 1.9 milliards.
Dans l’hypothèse 2 (prix taxé, 8,4\euro) on obtient un chiffre d’affaire de 1.7 milliards, qui génère 1,3 milliard à 80\% de taxes, en faisant l’hypothèse d’une éviction du marché noir. La \textbf{table \ref{tab:synthese2}} synthétise ces résultats.


 \begin{table}[p]\centering
 \hspace{-2cm}\begin{tabular}{p{3.8cm}|cccccc}
  &Statu quo& Légalisation		& Variation & Légalisation	& Variation \\
  &   		& prix de vente		&			& prix de vente		&			\\
  &  		& inchangé (6\euro)	&			& majoré (8\euro 40)& 			\\\hline \hline   
  Usagers quotidiens (en milliers)  & 550  	&812  	&+47.6 \%  	&550  	&0.0 \% \\\hline
  Volume trafic  (en tonnes)  		& 277  	&457  	&+65.0 \%  	&277  	&0.0 \% \\\hline
  Prix de vente (en euros)  		& 6  	&6  	&0.0 \% 	& 8.4  	&+40.0 \%\\\hline
  Coût d’acquisition (en euros) 	& 8.4	&6 		& -28.6 \%  &8.4  	&0.0\%\\\hline
  Dépenses publiques (en millions d’euros) 	& 568  	&65.8       &-88.4 \%  &44.6  &-92.1 \%\\ \hline
  Recettes publiques (en millions d’euros)	&0		& 1647	&& 1331 &\\\hline
 \end{tabular}
 \label{tab:synthese2}
 \caption{Tableau de synthèse du scénario 2 \cite{terraNova_rapport}}
\end{table} 
 
 
\section{Scénario 3 : Légaliser l’usage et la vente dans un cadre concurrentiel}
 
Ce scénario ressemble beaucoup au scénario 2, excepté une grosse différence : au lieu de s’inscrire dans le cadre d’un monopole public avec un prix fixé par l’Etat, le prix est décidé cette fois par le marché concurrentiel.
Ce scénario laisse imaginer que le prix du cannabis descendra rapidement à cause de la concurrence, ce qui entraînera une forte hausse de la consommation.
Pour une baisse de prix de vente de 10\%, Terra Nova estime que la consommation en tonnage se verra doubler (544 tonnes au lieu de 274tonnes) et le nombre de consommateur quotidien augmentera de 393 000, soit 71\%.
L’État collecterait 1.7milliards d’euros  de taxes et obtiendrait un gain budgétaire (économie sur les dépenses publiques + gains des taxes) de 2.2 milliards d’euros.
Ce scénario a l’avantage de pouvoir supprimer rapidement les réseaux clandestins et de générer de fortes recettes. Cependant, le nombre de consommateurs ainsi que la consommation augmentant, ce scénario entraînerait des dommages sanitaires massifs. La \textbf{table \ref{tab:synthese3}} synthétise ces résultats.
 

\begin{table}[p]\centering
 \begin{tabular}{p{3.8cm}|cccccc}
  &Statu quo& Légalisation		& Variation  \\
  &   		& dans un cadre		&						\\
  &  		& concurrentiel	&			 			\\\hline \hline   
  Usagers quotidiens (en milliers)  & 550  	&943  	&+71.5 \%  	 \\\hline
  Volume trafic  (en tonnes)  		& 277  	&544  	&+96.4 \%  	 \\\hline
  Prix de vente (en euros)  		& 6  	&5.4&-10.0 \%  \%\\\hline
  Coût d’acquisition (en euros) 	& 8.4	&5.4 		&-35.7 \%  \\\hline
  Dépenses publiques (en millions d’euros) 	& 568  	&76       &-86.4 \%  \\ \hline
  Recettes publiques (en millions d’euros)	&0		& 1764	&\\\hline
 \end{tabular}
 \label{tab:synthese3}
 \caption{Tableau de synthèse du scénario 3 \cite{terraNova_rapport}}
\end{table} 
 
 
\section{Synthèse}

	Le scénario le plus adapté est donc le scénario 2 qui est le seul à permettre une régulation du nombre de consommateurs de cannabis, tout en générant des profits pour l'Etat ainsi qu'un gisement d'emplois.
